\documentclass{article}
\usepackage{tikz}
\usetikzlibrary{graphs,graphdrawing}
\usepackage{multirow}
\usepackage{multicol}
\usepackage{tkz-graph}
\usegdlibrary{trees}


\usepackage[margin=2cm]{geometry}

\usepackage[utf8]{inputenc}
\begin{document}
	\section{Esercitazione 1}
	\subsection{Esercizio 1}
	
	Dimostrare che \( P(a | b \land	a) = 1  \). \\
	
	Possiamo riscrivere  \( P(a | b \land	a)\) come 
	\begin{equation}
	P(a | b \land	a) = \frac{ P(a \land b \land a)}{P(b \land a)}
	\end{equation}

	Questo per la definizione di probabilità condizionata che dice che:
	\[ P(A | B) = \frac{ P(A \land B )}
					    {P(B)}\]
					    
	Sappiamo anche che \(P(a \land b \land a)\) = \(P(a \land b)\), per cui possiamo riscrivere l'equazione 1 come
	
	\[P(a | b \land	a) = \frac{ P(a \land b)}{P(b \land a)}\] e dato che \(P(a \land b)\) = \(P(b \land a)\) possiamo concludere che \( P(a | b \land	a) = 1  \).
	
	
	\subsection{Esercizio 2}
	Date le seguenti belief di un agente razionale: \\
	\\
	\( P(A) = 0.4 \) \hspace{2cm}   \( P(B) = 0.3 \) \hspace{2cm} 	\( P(A \lor B) = 0.5 \) \\
	
	Quale range di probabilità è ragionevole per  	\( A \land V \) ? \\
	
	Sapendo che 
	\[ P(A \lor B) = P(A) + P(B) - P(A \land B) \] 
	y
	\[0.4 + 0.3 - P(A \land B) = 0.5 \]
	
	Da cui:
	
	\[P(A \land B) = 0.2 \]
	\pagebreak
	\subsection{Esercizio 3}
	Supponiamo di conoscere che la probabilità con cui un
	semaforo diventi verde sia pari a 0.45, arancione pari a
	0.1 e che diventi rosso sia pari a 0.45.
	Inoltre, supponiamo di avere la probabilità del 25\% di
	passare con il semaforo rosso senza prendere una
	multa , e il 5\% di probabilità di prendere una multa
	passando con il semaforo arancione
	
	In aggiunta, supponiamo che prendendo una multa, c’è
	il 90\% di probabilità che si sia successivamente di
	cattivo umore; senza multa la probabilità è del 5\%.
	
	Qual è la probabilità totale di essere di cattivo umore?
	\\
	\vspace{2cm}
	\\
	\begin{center}
	\begin{tikzpicture}
		\graph[tree layout,grow=right,fresh nodes] {
			"" -- {
				"green (0.45)" -- {
					"ticket (0)" -- {
						"bad-mood (0.9)",
						"good mood (0.1)" 
					},
					"not ticket (1)" -- {
						"bad-mood (0.05)",
						"good mood (0.95)" 
					},  
				},
				"yellow (0.1)" -- {
					"ticket (0.05)" -- {
						"bad-mood (0.9)",
						"good mood (0.1)" 
					},
					"not ticket (0.95)" -- {
						"bad-mood (0.05)",
						"good mood (0.95)" 
					},  
				},
				"red (0.45)" -- {
					"ticket (0.25)" -- {
						"bad-mood (0.9)",
						"good mood (0.1)" 
					},
					"not ticket (0.75)" -- {
						"bad-mood (0.05)",
						"good mood (0.95)" 
					},  
				},
			}
		};
	\end{tikzpicture}
	\end{center}		    

Possiamo costruire la matrice delle probabilità congiunte:
\begin{center}
	\begin{tabular}{ |c|c|c|c|c|c|c| } 
		\hline
		 & \multicolumn{2}{|c|}{Green} & \multicolumn{2}{|c|}{Yellow} & \multicolumn{2}{|c|}{Red} 
		\\ \hline
		& Ticket & \lnot Ticket & Ticket &  \lnot Ticket & Ticket & \lnot Ticket 
		\\ \hline
		Bad  & 0.45 * 0 * 0.9 &  0.45 * 1 * 0.05  & 0.1 * 0.05 *0.9  & 0.1 * 0.95 *0.05  & 0.45 * 0.25 *0.9 &  0.45 * 0.75 * 0.05
		\\ \hline
		Good & 0.45 * 0 * 0.1 &  0.45 * 1 * 0.95  & 0.1 * 0.05 *0.1  & 0.1 * 0.95 *0.95  & 0.45 * 0.25 *0.1 &  0.45 * 0.75 * 0.95
		\\ \hline
	\end{tabular}
\end{center}

Da cui:

\begin{center}
	\begin{tabular}{ |c|c|c|c|c|c|c| } 
		\hline
		& \multicolumn{2}{|c|}{Green} & \multicolumn{2}{|c|}{Yellow} & \multicolumn{2}{|c|}{Red} 
		\\ \hline
		& Ticket & \lnot Ticket & Ticket &  \lnot Ticket & Ticket & \lnot Ticket 
		\\ \hline
		Bad  & 0  & 0.0225  &  0.0045  & 0.00475  & 0.10125  &  0.016875
		\\ \hline
		Good & 0  & 0.4275  &  0.00005 & 0.09025  & 0.01125  &  0.320625
		\\ \hline
	\end{tabular}
\end{center}

\pagebreak

\subsection{Esercizio 4}
Una società di consulenza ha creato un modello per prevedere le recessioni. Il modello
prevede una recessione con l’80\% di probabilità quando la recessione avviene realmente
e con il 10\% di probabilità quando non avviene. La probabilità incondizionata che si entri
in una fase di recessione è del 20\%. Se il modello prevede la recessione, qual è la
probabilità che la recessione avvenga?
\\  
\\
Sappiamo che  \\

\[ P(rec. \; pred. \;|\; rec. \; coming) = \frac{8}{10} \] ,che  \[ P(rec. pred. \;|\; rec. \; not  \; coming) = \frac{1}{10} \] e che  \( P(rec.\; coming) = \frac{2}{10} \) \\

Vogliamo trovare \( P(rec.\; coming \;|\; rec.\; pred)  \) \\

Per la regola di Bayes sappiamo che :

\[
	P(rec.\; coming \;|\; rec.\; pred) = \frac{	P(rec.\; pred \;|\;  rec.\; coming ) * P(rec.\; coming )}{P(rec. \; pred)}
\]

Qual'è la \(P(rec. \; pred)\) ? \\

Sappiamo che 
\[
P(rec. \; pred) = P(rec. \; pred \;|\; rec. coming) * P(rec. \; coming) + 
P(rec. \; pred \;|\; rec. \; not \; coming) * P(rec. \; not \; coming) 
\]

Inoltre se  \( P(rec.\; coming) = \frac{2}{10} \) segue che \( P(rec.\; not \; coming) = \frac{8}{10} \) \\

Da cui:

\[
P(rec. \; pred) =  \frac{8}{10} \cdot \frac{2}{10} + \frac{1}{10}\cdot \frac{8}{10} = \frac{24}{100}
\]

Concludendo quindi che :

\[
P(rec.\; coming \;|\; rec.\; pred) = \frac{	\frac{8}{10} \cdot	\frac{2}{10}}{\frac{24}{100}} = \frac{\frac{16}{100}}{\frac{24}{100}} =
{\frac{16}{100}} \cdot {\frac{100}{24}} = \frac{2}{3}
\]

\pagebreak

\subsection{Esercizio 5}
Sviluppare una rete di Bayes, per calcolare la probabilità che uno studente superi l’esame di MPD. \\ Le proprietà di interesse del problema sono:

\begin{itemize}
	\item Il superamento dell’esame \( EX \in \{true, false\} \ \)
	\item L’acquisizione di buone capacità pratiche in MPD da parte dello studente \( PR \in \{true, false\} \ \)
	\item L’acquisizione di buone capacità teoriche in MPD da parte dello studente \( TE \in \{true, false\} \ \)
	\item Lo studente è efficientemente studioso \( ST \in \{true, false\} \ \)
	\item La quantità di esercitazioni seguite dallo studente \( QE \in \{molte, poche, nessuna\} \ \)
	\item L’aver fatto un numero sufficiente di esercizi \( SE \in \{true, false\} \ \)
	
\end{itemize}

Costruire una rete bayesiana che rappresenti la conoscenza probabilistica relativa
al dominio descritto dalle seguenti relazioni di dipendenza tra le variabili casuali:

\begin{itemize}
	\item Il superamento dell’esame dipende dalle capacità
	teoriche e pratiche dello studente
	\item Se uno studente è studioso ha buone probabilità di acquisire
	capacità teoriche.
	\item Il numero di esercitazioni seguite dipende da quanto uno
	studente è studioso
	\item L’aver fatto sufficienti esercizi dipende dal numero di
	esercitazioni seguite, ed influenza le capacità pratiche dello
	studente.
\end{itemize}

Dobbiamo un creare un DAG a partire dai sequenti nodi: \\

\begin{center}
	\begin{tikzpicture}
		\SetUpEdge[lw         = 1.5pt,
		color      = orange,
		labelcolor = white]
		\GraphInit[vstyle=Normal] 
		\SetGraphUnit{3}
		\tikzset{VertexStyle/.append  style={fill}}
		\Vertex{QE}
		\NOEA(QE){SE}  \EA(SE){PR} \SOEA(PR){EX}
		\SOEA(QE){ST}  \EA(ST){TE}
	\end{tikzpicture}
\end{center}
\pagebreak

Sappiamo che il superamento dell'esame \(EX\) dipende  dalle capacità teoriche (\(TE\)) e pratiche (\(PR\)) dello studente	

\begin{center}
	\begin{tikzpicture}
		\SetUpEdge[lw         = 1.5pt,
		color      = orange,
		labelcolor = white]
		\GraphInit[vstyle=Normal] 
		\SetGraphUnit{3}
		\tikzset{VertexStyle/.append  style={fill}}
		\Vertex{QE}
		\NOEA(QE){SE}  \EA(SE){PR} \SOEA(PR){EX}
		\SOEA(QE){ST}  \EA(ST){TE}
		 \tikzset{EdgeStyle/.style={->}}
		\Edge(PR)(EX)
		\Edge(TE)(EX)
	\end{tikzpicture}
\end{center}		

Sappiamo anche che se uno studente è studioso  (\(ST\)) ha buone probabilità di acquisire capacità teoriche  (\(TE\)).

\begin{center}
	\begin{tikzpicture}
		\SetUpEdge[lw         = 1.5pt,
		color      = orange,
		labelcolor = white]
		\GraphInit[vstyle=Normal] 
		\SetGraphUnit{3}
		\tikzset{VertexStyle/.append  style={fill}}
		\Vertex{QE}
		\NOEA(QE){SE}  \EA(SE){PR} \SOEA(PR){EX}
		\SOEA(QE){ST}  \EA(ST){TE}
		\tikzset{EdgeStyle/.style={->}}
		\Edge(PR)(EX)
		\Edge(TE)(EX)
		\Edge(ST)(TE)
	\end{tikzpicture}
\end{center}

Sappiamo anche che la quantità di esercitazioni seguite  (\(TE\))  dipende da quanto lo studente sia studioso  (\(ST\))

\begin{center}
	\begin{tikzpicture}
		\SetUpEdge[lw         = 1.5pt,
		color      = orange,
		labelcolor = white]
		\GraphInit[vstyle=Normal] 
		\SetGraphUnit{3}
		\tikzset{VertexStyle/.append  style={fill}}
		\Vertex{QE}
		\NOEA(QE){SE}  \EA(SE){PR} \SOEA(PR){EX}
		\SOEA(QE){ST}  \EA(ST){TE}
		\tikzset{EdgeStyle/.style={->}}
		\Edge(PR)(EX)
		\Edge(TE)(EX)
		\Edge(ST)(TE)
		\Edge(ST)(QE)
	\end{tikzpicture}
\end{center}

Inioltre sappiamo è che avere fatto un numero sufficente di esercizi (\(SE\)) dipende dalla quantità di esercitazioni seguite (\(QE\))

\begin{center}
	\begin{tikzpicture}
		\SetUpEdge[lw         = 1.5pt,
		color      = orange,
		labelcolor = white]
		\GraphInit[vstyle=Normal] 
		\SetGraphUnit{3}
		\tikzset{VertexStyle/.append  style={fill}}
		\Vertex{QE}
		\NOEA(QE){SE}  \EA(SE){PR} \SOEA(PR){EX}
		\SOEA(QE){ST}  \EA(ST){TE}
		\tikzset{EdgeStyle/.style={->}}
		\Edge(PR)(EX)
		\Edge(TE)(EX)
		\Edge(ST)(TE)
		\Edge(ST)(QE)
		\Edge(QE)(SE)
	\end{tikzpicture}
\end{center}

Infine sappiamo che avere svolto un numero sufficiente di esercizi influenza le capacità pratiche dello studente.

\begin{center}
	\begin{tikzpicture}
		\SetUpEdge[lw         = 1.5pt,
		color      = orange,
		labelcolor = white]
		\GraphInit[vstyle=Normal] 
		\SetGraphUnit{3}
		\tikzset{VertexStyle/.append  style={fill}}
		\Vertex{QE}
		\NOEA(QE){SE}  \EA(SE){PR} \SOEA(PR){EX}
		\SOEA(QE){ST}  \EA(ST){TE}
		\tikzset{EdgeStyle/.style={->}}
		\Edge(PR)(EX)
		\Edge(TE)(EX)
		\Edge(ST)(TE)
		\Edge(ST)(QE)
		\Edge(QE)(SE)
		\Edge(SE)(PR)
	\end{tikzpicture}
\end{center}
\pagebreak
 
 Abbiamo ora la struttura topologica, è neccessario avere anche le distribuzione di probabilità a priori o condizionate.

Il nodo (\(ST\)) ha una distribuzione di proprità a priori per i suoi possibili valori (\({true, false}\)).

Tutte le altre probabilità sono condizionate da uno o più genitori:

Prendiamo dei dati a caso come esempio:
\\

\begin{tabular}{|ll|}
	\hline
	\multicolumn{2}{|l|}{\(P(ST) \)}         \\ \hline
	\multicolumn{1}{|l|}{\(true\)} & \(false\) \\ \hline
	\multicolumn{1}{|l|}{0.7}  & 0.3   \\ \hline
\end{tabular}



Per quando riguarda ad esempio la probabilità \(TE\)  è condizionata da un solo genitore \(ST\), avremo quindi una situazione di questo tipo:
\\

\begin{tabular}{|l|ll|}
	\hline
	ST    & \multicolumn{2}{l|}{P(TE |ST)} \\ \hline
	& \multicolumn{1}{l|}{T}   & F   \\ \hline
	true  & \multicolumn{1}{l|}{x}   & 1-x    \\ \hline
	false & \multicolumn{1}{l|}{y}   & 1-y   \\ \hline
\end{tabular}

Se prendiamo una probabilità con due genitori, come \(EX\), avremo una situazione di questo tipo:
\\

\begin{tabular}{|ll|ll|}
	\hline
	\multicolumn{2}{|l|}{}              & \multicolumn{2}{l|}{P(EX |TE, PR)} \\ \hline
	\multicolumn{1}{|l|}{TE}    & PR    & \multicolumn{1}{l|}{true}  & false \\ \hline
	\multicolumn{1}{|l|}{true}  & true  & \multicolumn{1}{l|}{x}     &       \\ \hline
	\multicolumn{1}{|l|}{false} & false & \multicolumn{1}{l|}{y}     &       \\ \hline
	\multicolumn{1}{|l|}{false} & true  & \multicolumn{1}{l|}{z}     &       \\ \hline
	\multicolumn{1}{|l|}{false} & false & \multicolumn{1}{l|}{a}     &       \\ \hline
\end{tabular}

Proviamo a costruire la tabella della probabilità \(QE\) con dei valori plausibili, un esempio potrebbe essere:
\\

\begin{tabular}{|c|ccc|}
	\hline
	ST    & \multicolumn{3}{c|}{P(EX |TE, PR)}                                \\ \hline
	& \multicolumn{1}{c|}{molte} & \multicolumn{1}{c|}{poche} & nessuna \\ \hline
	true  & \multicolumn{1}{c|}{0.5}   & \multicolumn{1}{c|}{0.3}   & 0.2     \\ \hline
	false & \multicolumn{1}{c|}{0}     & \multicolumn{1}{c|}{0.1}   & 0.9     \\ \hline
\end{tabular}
  
 \pagebreak
 
 \subsection{Esercizio 6}
 Consideriamo questa rete bayesiana:
 \\
 \begin{center}
 	\begin{tikzpicture}
 		\SetUpEdge[lw         = 1.5pt,
 		color      = orange,
 		labelcolor = white]
 		\GraphInit[vstyle=Normal] 
 		\SetGraphUnit{3}
 		\tikzset{VertexStyle/.append  style={fill}}
 		\Vertex{A}
 		\tikzset{EdgeStyle/.style={->}}
 		\SOWE(A){C}  \SO(A){D} \SOEA(A){F}
 		\SO(C){B}    \SO(D){E}
 		\Edge(A)(C)
 		\Edge(A)(D)
 		\Edge(A)(F)
 		\Edge(C)(D)
 		\Edge(F)(E)
 		\Edge(D)(E)
 		\Edge(B)(C)
 		\Edge(B)(E)
 	\end{tikzpicture}
 \end{center}

Quali sono le probabilità da definire per avere una rete bayesiana?
\\

\begin{itemize}
	\item \( P(A) \)
	\item \( P(B) \)
	\item \( P(C | B, A) \)
	\item \( P(D | C, A) \)
	\item \( P(E | B,D,F) \)
	\item \( P(F | A) \)
\end{itemize}

\pagebreak

 \subsection{Esercizio 7}
  Un paziente si reca dal dottore per sottoporre una patologia, il dottore sospetta 3 possibili malattie come
 causa della condizione patologica. Le 3 malattie sono \(D_1, D_2, D_3 \) le quali sono marginalmente
 indipendenti tra loro. Ci sono 4 sintomi \(S_1, S_2, S_3, S_4 \) di cui il dottore vuole verificare la presenza in modo
 da trovare la causa più probabile per la condizione patologica. I sintomi sono condizionalmente dipendenti
 alle 3 malattie come segue: \(S_1\) dipende solamente da \(D_1\), \(S_2\) dipende da \(D_1\) e da \(D_2\) , \(S_3\) dipende da \(D_1\) e da \(D_3\), e \(S_4\) dipende solamente da \(D_3\). Si assuma che tutte le variabili casuali siano booleane.
 
 \begin{itemize}
 	\item Costruire la struttura topologica alla rete bayesiana per il problema descritto.
\end{itemize}

 \begin{center}
	\begin{tikzpicture}
		\SetUpEdge[lw         = 1.5pt,
		color      = orange,
		labelcolor = white]
		\GraphInit[vstyle=Normal] 
		\SetGraphUnit{3}
		\tikzset{VertexStyle/.append  style={fill}}
		\Vertex{D1}
		\EA(D1){D2}
		\EA(D2){D3}
		\SOEA(D1){S1}
		\EA(S1){S2}
		\EA(S2){S3}
		\EA(S3){S4}
		\tikzset{EdgeStyle/.style={->}}
		\Edge(D1)(S1)
		\Edge(D1)(S2)
		\Edge(D2)(S2)
		\Edge(D3)(S3)
		\Edge(D1)(S3)
		\Edge(D3)(S4)
	\end{tikzpicture}
\end{center}

 \begin{itemize}
	\item Scrivere la distribuzione di probabilità congiunta espressa come prodotto delle probabilità condizionate
\end{itemize}

\[P(D_1, D_2, D_3, S_1, S_2, S_3, S_4) = \]
\[= P(D_1) \cdot P(D_2) \cdot P(D_3) \cdot    \]
\[= P(S_1 \;|\; D_1) \cdot  P(S_2 \;|\; D_1, D_2)\cdot    \]
\[= P(S_3 \;|\; D_1, D_3) \cdot  P(S_4 \;|\; D_3)  \]


 \begin{itemize}
	\item Qual è il numero di parametri indipendenti richiesti per descrivere la distribuzione congiunta?
\end{itemize}

Iniziamo ad immaginare le CPT (\textit{conditional probability table}) di ogni nodo.

Ad esempio per \(D_1\) avremo due valori nella CPT (\(true\; e\; false\)), così come  \(D_2\) e  \(D_3\).

Invece  \(S_1\) ha un genitore  ( \(D_1\)) per cui la tabella avrà quattro possibili valori.

Analogamente  \(S_2\) ha due  ( \(D_1\) e \(D_2\)) per cui la tabella avrà otto possibili valori.

Seguendo con questo ragionamento arriveremo ad avere
\[2 + 2 + 2 + 4 + 8 + 8 + 4 = 30\] parametri.

Come possiamo stimare i parametri indipendenti?

Essendo una distribuzione booleana per \(D_1\), \(D_2\) e \(D_3\) serve un parametro per ciascuno.

Per  \(S_1\) ne serviranno 2, così come per \(S_4\): per \(S_2\) ed \(S_3\) serviranno 4 variabili, in totale serviranno quindi 15 parametri.

 \begin{itemize}
	\item Assumendo che non ci sia dipendenza condizionale tra le variabili, quanti parametri indipendenti sarebbero
	dunque richiesti?
\end{itemize}

\[2^7 -1\]

\pagebreak

\subsection{Esercizio 8}

Supponiamo di avere la seguente rete Bayesiana su spazio booleano.

 \begin{center}
	\begin{tikzpicture}
		\SetUpEdge[lw         = 1.5pt,
		color      = orange,
		labelcolor = white]
		\GraphInit[vstyle=Normal] 
		\SetGraphUnit{3}
		\tikzset{VertexStyle/.append  style={fill}}
		\Vertex{A}
		\EA(A){B}
		\EA(B){C}
		\SOEA(A){D}
		\EA(D){E}
		\tikzset{EdgeStyle/.style={->}}
		\Edge(A)(D)
		\Edge(B)(D)
		\Edge(B)(E)
		\Edge(C)(E)
	\end{tikzpicture}
\end{center}

Sappiamo che \(P(A \; = \; T) = 0.2 \), \(P(B \; = \; T) = 0.2 \) e che \(P(C \; = \; T) = 0.8 \)


Sappiamo inoltre che:

\begin{tabular}{|c|c|c|}
	\hline
	A & B & P(D = T | A,B) \\ \hline
	T & T & 0.1        \\ \hline
	T & F & 0.5        \\ \hline
	F & T & 0.6        \\ \hline
	F & F & 0.9        \\ \hline
\end{tabular}

 e che  \\
\begin{tabular}{|c|c|c|}
 	\hline
 	C & B & P(E = T | C,B) \\ \hline
 	T & T & 0.3        \\ \hline
 	T & F & 0.8        \\ \hline
 	F & T & 0.4        \\ \hline
 	F & F & 0.2        \\ \hline
\end{tabular}

\begin{itemize}
	\item Qual è la probabilità che tutte le variabili siano false?
\end{itemize}


\[ P(A = F, B = F, C =F, D = F, E = F) = \] 

\[ = P(A = F) \cdot P(B = F) \cdot P(C = F) \cdot  P(D = F | A=F, B = F) \cdot P(E = F | B=F, C = F) \] 

\[ = 0.6 \cdot 0.5 \cdot 0.2 \cdot 0.1 \cdot 0.8 = 0.064 \] 

\begin{itemize}
	\item Qual è la probabilità di A, avendo la conoscenza che tutte le altre variabili sono vere?
\end{itemize}

Formalmente la richiesta è

\[ P(A  |B = T, C = T, D = T, E = T)\] che corrisponde all'unione di  \[ P(A = T |B = T, C = T, D = T, E = T)\] e \[ P(A = F |B = T, C = T, D = T, E = T)\]

https://elearning.unimib.it/pluginfile.php/1002302/mod\_resource/content/2/Esercitazione\%202.pdf
\
\end{document}

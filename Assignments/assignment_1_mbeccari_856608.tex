\documentclass{article}
\usepackage{tikz}
\usetikzlibrary{graphs,graphdrawing}
\usepackage{multirow}
\usepackage{multicol}
\usepackage{amsmath}

\usepackage{tkz-graph}
\usegdlibrary{trees}


\usepackage[margin=2cm]{geometry}

\usepackage[utf8]{inputenc}
\begin{document}
	{Michele Beccari 856608 - Modelli Probabilistici per le Decisioni -  Assignment 1 - Aprile 2023} 
	\section{Assignment 1}
Supponiamo di avere 3 monete truccate, denotate  a, b e c, in una borsa.
	
La probabilità che esca "testa" lanciando a, b e c è del 20\%, 60\% e 80\%, rispettivamente.
	
Estraggo una moneta dalla borsa (la probabilità di estrarre a, b o c è la stessa) e la lancio 3 volte.

\begin{itemize}
	\item Disegnare la Rete Bayesiana corrispondente (specificando le CPT). Caricare un'immagine o un documento contenente il disegno della Rete.
\end{itemize}

Per quanto detto al punto uno il gioco è rappresentato dalle seguenti variabili:

\begin{itemize}
	\item \(MP\) la moneta pescata dalla borsa
	\item \(R1\) il risultato del primo lancio
	\item \(R2\) il risultato del secondo lancio
	\item \(R3\) il risultato del terzo lancio
\end{itemize}

I tre lanci della moneta sono indipendenti fra di loro (il fatto che esca testa o croce al primo lancio non influenza il risultato dei successivi lanci) , mentre ogni lancio è dipendente da quale moneta è stata estratta.

 \begin{center}
	\begin{tikzpicture}
		\SetUpEdge[lw         = 1.5pt,
		color      = orange,
		labelcolor = white]
		\GraphInit[vstyle=Normal] 
		\SetGraphUnit{3}
		\tikzset{VertexStyle/.append  style={text = black}}
		\Vertex{MP}
		\tikzset{EdgeStyle/.style={->}}
		\SO(MP){R2} \EA(R2){R3} \WE(R2){R1}
		\Edge(MP)(R1)
		\Edge(MP)(R2)
		\Edge(MP)(R3)
	\end{tikzpicture}
\end{center}
	
Dopo avere definito la topologia della rete possiamo andare a definire le CPT. \\

{\renewcommand{\arraystretch}{1.5}%
\begin{tabular}{|ccc|}
	\hline
	\multicolumn{3}{|c|}{\(P(MP)\)}                          \\ \hline
	\multicolumn{1}{|c|}{a} & \multicolumn{1}{c|}{b} & c \\ \hline
	\multicolumn{1}{|c|}{$\frac{1}{3}$} & \multicolumn{1}{c|}{$\frac{1}{3}$} & {$\frac{1}{3}$} \\ \hline
\end{tabular}} \quad

\vspace{0.3cm}

\begin{minipage}[t]{0.3\textwidth}
	{\renewcommand{\arraystretch}{1.5}%
	\begin{tabular}{|c|cc|}
		\hline
		& \multicolumn{2}{c|}{\(P(R1 | MP)\)}  \\ \hline
		MP & \multicolumn{1}{c|}{T}    & C    \\ \hline
		a  & \multicolumn{1}{c|}{$\frac{2}{10}$} & $\frac{8}{10}$ \\ \hline
		b  & \multicolumn{1}{c|}{$\frac{6}{10}$} & $\frac{4}{10}$ \\ \hline
		c  & \multicolumn{1}{c|}{$\frac{8}{10}$} & $\frac{2}{10}$ \\ \hline
	\end{tabular}} \quad
\end{minipage}
\begin{minipage}[t]{0.3\textwidth}
		{\renewcommand{\arraystretch}{1.5}%
		\begin{tabular}{|c|cc|}
			\hline
			& \multicolumn{2}{c|}{\(P(R2 | MP)\)}  \\ \hline
			MP & \multicolumn{1}{c|}{T}    & C    \\ \hline
			a  & \multicolumn{1}{c|}{$\frac{2}{10}$} & $\frac{8}{10}$ \\ \hline
			b  & \multicolumn{1}{c|}{$\frac{6}{10}$} & $\frac{4}{10}$ \\ \hline
			c  & \multicolumn{1}{c|}{$\frac{8}{10}$} & $\frac{2}{10}$ \\ \hline
	\end{tabular}} \quad
\end{minipage}
\begin{minipage}[t]{0.3\textwidth}
		{\renewcommand{\arraystretch}{1.5}%
		\begin{tabular}{|c|cc|}
			\hline
			& \multicolumn{2}{c|}{\(P(R3 | MP)\)}  \\ \hline
			MP & \multicolumn{1}{c|}{T}    & C    \\ \hline
			a  & \multicolumn{1}{c|}{$\frac{2}{10}$} & $\frac{8}{10}$ \\ \hline
			b  & \multicolumn{1}{c|}{$\frac{6}{10}$} & $\frac{4}{10}$ \\ \hline
			c  & \multicolumn{1}{c|}{$\frac{8}{10}$} & $\frac{2}{10}$ \\ \hline
	\end{tabular}} \quad
\end{minipage}


\newpage

\begin{itemize}
	\item Calcolare la probabilità che avendo ottenuto Testa, Testa, Croce la moneta estratta sia la a.
\end{itemize}

Per soddisfare la richiesta dobbiamo calcolare la distribuzione di probabilità congiunta sapendo che :

\[
	P(A | R1 \;=\; T,\;  R2 \;=\; T,\; R3 \;=\; C) \; \lor \; P(B | R1 \;=\; T,\;  R2 \;=\; T,\; R3 \;=\; C) \; \lor \; P(B | R1 \;=\; T,\;  R2 \;=\; T,\; R3 \;=\; C) = 1
\]

Procediamo quindi con il calcolo delle probabilità delle singole monete per la costante di normalizzazione \(\alpha\):

\[
P(A | R1 \;=\; T,\;  R2 \;=\; T,\; R3 \;=\; C) = 
\]

\[
= \alpha \cdot P(A) \cdot  P(R1 = T|A) \cdot  P(R2 = T|A)  \cdot  P(R3 = C|A) = 
\]

Essendo i tre lanci di monete indipendenti sappiamo che:
\[
 P(R1 = T|A) =  P(R2= T|A)
\]

da cui 

\[
= \alpha \cdot P(A) \cdot  P(R1 = T|A)^2 \cdot P(R3= C|A)  = 
\]

\[
= \alpha \cdot \frac{1}{3} \cdot  (\frac{2}{10})^2 \cdot \frac{8}{10}  = 
\]

\[
= \alpha \cdot \frac{4}{375}
\]

Per B avremo:

\[
P(B | R1 \;=\; T,\;  R2 \;=\; T,\; R3 \;=\; C) = 
\]

\[
= \alpha \cdot P(B) \cdot  P(R1 = T|B) \cdot  P(R2 = T|B)  \cdot  P(R3 = C|B) = 
\]

Essendo i tre lanci di monete indipendenti sappiamo che:
\[
P(R1|B) =  P(R2|B)
\]

da cui:

\[
= \alpha \cdot P(B) \cdot  P(R1 = T|B)^2  \cdot P(R3 = C|B) = 
\]

\[
= \alpha \cdot \frac{1}{3} \cdot (\frac{6}{10})^2 \cdot (\frac{4}{10})   = 
\]


\[
= \alpha \cdot \frac{6}{125}
\]

Seguendo lo stesso ragionamento per \(C\):

\[
= \alpha \cdot P(C) \cdot  P(R1 = T|C)^2 \cdot  P(R3 = C|C) = 
\]

\[
= \alpha \cdot \frac{1}{3} \cdot  (\frac{8}{10})^2 \cdot \frac{2}{10}  = 
\]

\[
= \alpha \cdot \frac{16}{375}
\]

\pagebreak

A questo punto possiamo calcolare la costante di normalizzazione \(\alpha\)

\[
\alpha = \frac{1}
{
	\frac{4}{375} + \frac{6}{125} + \frac{16}{375}
} = 
\frac{375}{38}
\]

Da cui concludiamo che
\[
P(A | R1 \;=\; T,\;  R2 \;=\; T,\; R3 \;=\; C) = 
\]

\[
 = \alpha \cdot \frac{4}{375} =  \frac{375}{38} \cdot \frac{4}{375} \approx 0.1053
\]

\pagebreak

\begin{itemize}
	\item Data la seguente Rete Bayesiana, si risponda (vero o falso) se le seguenti affermazioni di indipendenza condizionale sono riflesse dalla struttura della rete
\end{itemize}

 \begin{center}
	\begin{tikzpicture}
		\SetUpEdge[lw         = 1.5pt,
		color      = orange,
		labelcolor = white]
		\GraphInit[vstyle=Normal] 
		\SetGraphUnit{3}
		\tikzset{VertexStyle/.append  style={fill}}
		\Vertex{S}
		\tikzset{EdgeStyle/.style={->}}
		\SO(S){W}  \WE(W){R}
		\SO(R){A} \SO(S){W}  \EA(W){J}
		\SO(J){F} \SO(W){D}
		\Edge(S)(R)
		\Edge(S)(W)
		\Edge(S)(J)
		\Edge(W)(R)
		\Edge(W)(J)
		\Edge(R)(A)
		\Edge(R)(D)
		\Edge(A)(D)
		\Edge(D)(F)
		\Edge(J)(F)
	\end{tikzpicture}
\end{center}

\begin{itemize}
	\item R e J d-separano F e S?
\end{itemize}

 \begin{center}
	\begin{tikzpicture}
		\SetUpEdge[lw         = 1.5pt,
		color      = orange,
		labelcolor = white]
		\GraphInit[vstyle=Normal] 
		\SetGraphUnit{3}
		\tikzset{VertexStyle/.append  style={fill = white} }
		\Vertex{S}
		\tikzset{EdgeStyle/.style={->}}
		\SO(S){W} 
		\tikzset{VertexStyle/.append  style={fill = gray, text=white} }
 		\WE(W){R} \EA(W){J}
 		\tikzset{VertexStyle/.append  style={fill = white, text=black} }
		\SO(R){A} \SO(S){W}  
		\SO(J){F} \SO(W){D}
		\Edge(S)(R)
		\Edge(S)(W)
		\Edge(S)(J)
		\Edge(W)(R)
		\Edge(W)(J)
		\Edge(R)(A)
		\Edge(R)(D)
		\Edge(A)(D)
		\Edge(D)(F)
		\Edge(J)(F)
	\end{tikzpicture}
\end{center}

R e J d-separano F e S perchè sono presenti quattro casi di archi di "tail-to-head" (S -> R -> D)  (S -> J -> F) (W -> J -> F) (S -> R -> A )  che bloccano tutti i cammini non orientati da F a S.
\pagebreak

\begin{itemize}
	\item J d-separa F e W?
\end{itemize}

 \begin{center}
	\begin{tikzpicture}
		\SetUpEdge[lw         = 1.5pt,
		color      = orange,
		labelcolor = white]
		\GraphInit[vstyle=Normal] 
		\SetGraphUnit{3}
		\tikzset{VertexStyle/.append  style={fill = white} }
		\Vertex{S}
		\tikzset{EdgeStyle/.style={->}}
		\SO(S){W} 
		\tikzset{VertexStyle/.append  style={fill = white, text=black} }
		\WE(W){R} 
		\tikzset{VertexStyle/.append  style={fill = gray, text=white} }
		\EA(W){J}
		\tikzset{VertexStyle/.append  style={fill = white, text=black} }
		\SO(R){A} \SO(S){W}  
		\SO(J){F} \SO(W){D}
		\Edge(S)(R)
		\Edge(S)(W)
		\Edge(S)(J)
		\Edge(W)(R)
		\Edge(W)(J)
		\Edge(R)(A)
		\Edge(R)(D)
		\Edge(A)(D)
		\Edge(D)(F)
		\Edge(J)(F)
	\end{tikzpicture}
\end{center}

J non  d-separa  F e W perché su un cammino non orientato tra F e W  non è presente nessun tipo di blocco (F -> D -> R -> W -> S)


\begin{itemize}
	\item W e R d-separano D e S?
\end{itemize}

 \begin{center}
	\begin{tikzpicture}
		\SetUpEdge[lw         = 1.5pt,
		color      = orange,
		labelcolor = white]
		\GraphInit[vstyle=Normal] 
		\SetGraphUnit{3}
		\tikzset{VertexStyle/.append  style={fill = white} }
		\Vertex{S}
		\tikzset{EdgeStyle/.style={->}}
		\tikzset{VertexStyle/.append  style={fill = gray, text=white} }
		\SO(S){W} 
		\WE(W){R}
		\tikzset{VertexStyle/.append  style={fill = white, text=black} }
		\EA(W){J}
		\SO(R){A}
		\SO(J){F} \SO(W){D}
		\Edge(S)(R)
		\Edge(S)(W)
		\Edge(S)(J)
		\Edge(W)(R)
		\Edge(W)(J)
		\Edge(R)(A)
		\Edge(R)(D)
		\Edge(A)(D)
		\Edge(D)(F)
		\Edge(J)(F)
	\end{tikzpicture}
\end{center}

W e R d-separano D e S perchè tutti i cammini non orientati sono bloccati: sono presenti tre  blocchi di tipo "tail-to-head" (S -> W -> J) (S -> R -> D) (S -> R -> A) e un blocco di tipo "head-to-head" (J -> F <- D)

\pagebreak
\begin{itemize}
	\item R d-separa D e J?
\end{itemize}

 \begin{center}
	\begin{tikzpicture}
		\SetUpEdge[lw         = 1.5pt,
		color      = orange,
		labelcolor = white]
		\GraphInit[vstyle=Normal] 
		\SetGraphUnit{3}
		\tikzset{VertexStyle/.append  style={fill = white} }
		\Vertex{S}
		\tikzset{EdgeStyle/.style={->}}
		\SO(S){W} 
		\tikzset{VertexStyle/.append  style={fill = gray, text=white} }
		\WE(W){R}
		\tikzset{VertexStyle/.append  style={fill = white, text=black} }
		\EA(W){J}
		\SO(R){A}
		\SO(J){F} \SO(W){D}
		\Edge(S)(R)
		\Edge(S)(W)
		\Edge(S)(J)
		\Edge(W)(R)
		\Edge(W)(J)
		\Edge(R)(A)
		\Edge(R)(D)
		\Edge(A)(D)
		\Edge(D)(F)
		\Edge(J)(F)
	\end{tikzpicture}
\end{center}


R d-separa D e J perchè sono presenti tre  blocchi di tipo "tail-to-head" (W -> R -> D) (S -> R -> A) (W -> R -> A) e un blocco di tipo "head-to-head" (J -> F <- D).


\begin{itemize}
 \item	Si indichino tutte le coppie di nodi che sono indipendenti l'uno dall'altro dati i nodi S, R e D.
\end{itemize}

 \begin{center}
	\begin{tikzpicture}
		\SetUpEdge[lw         = 1.5pt,
		color      = orange,
		labelcolor = white]
		\GraphInit[vstyle=Normal] 
		\SetGraphUnit{3}
		\tikzset{VertexStyle/.append  style={fill = white} }
		\tikzset{VertexStyle/.append  style={fill = gray, text=white} }
		\Vertex{S}
		\tikzset{VertexStyle/.append  style={fill = white, text=black} }
		\tikzset{EdgeStyle/.style={->}}
		\SO(S){W} 
		\tikzset{VertexStyle/.append  style={fill = gray, text=white} }
		\WE(W){R}
		\tikzset{VertexStyle/.append  style={fill = white, text=black} }
		\EA(W){J}
		\SO(R){A}
		\SO(J){F} 
		\tikzset{VertexStyle/.append  style={fill = gray, text=white} }
		\SO(W){D}
		\Edge(S)(R)
		\Edge(S)(W)
		\Edge(S)(J)
		\Edge(W)(R)
		\Edge(W)(J)
		\Edge(R)(A)
		\Edge(R)(D)
		\Edge(A)(D)
		\Edge(D)(F)
		\Edge(J)(F)
	\end{tikzpicture}
\end{center}


\end{document}



